\documentclass[11pt]{article}
\usepackage[margin=1in]{geometry}          
\usepackage{amsthm}
\usepackage{amsmath}
\usepackage{amssymb}
\usepackage{setspace}\onehalfspacing
\usepackage[loose,nice]{units} %replace "nice" by "ugly" for units in upright fractions
\usepackage{graphicx}
\graphicspath{ {/Users/eghuang/math104/images/} }
\usepackage{listings}
\usepackage{color}

\definecolor{dkgreen}{rgb}{0,0.6,0}
\definecolor{gray}{rgb}{0.5,0.5,0.5}
\definecolor{mauve}{rgb}{0.58,0,0.82}

\lstset{frame=tb,
  language=R,
  aboveskip=3mm,
  belowskip=3mm,
  showstringspaces=false,
  columns=flexible,
  basicstyle={\small\ttfamily},
  numbers=none,
  numberstyle=\tiny\color{gray},
  keywordstyle=\color{blue},
  commentstyle=\color{dkgreen},
  stringstyle=\color{mauve},
  breaklines=true,
  breakatwhitespace=true,
  tabsize=3
}

\title{
  Notes on Inference Devices \\
  \large Santa Fe Institute}
\author{Edward G. Huang}
\date{Summer 2018} 
 
 \newcommand{\R}{\mathbb{R}}
 \newcommand{\Q}{\mathbb{Q}}
 \newcommand{\Z}{\mathbb{Z}}
 \newcommand{\N}{\mathbb{N}}
 \newcommand{\B}{\mathbb{B}}
 \newcommand{\Prob}{\mathbb{P}}
 \newcommand{\E}{\mathbb{E}}
 \newcommand{\code}[1]{\texttt{#1}}
 \let\oldemptyset\emptyset
 \let\emptyset\varnothing
 
 \setlength{\parindent}{0pt} % no indent
 
\begin{document}
\maketitle 


\textbf{Notation and Definitions} \\
\\
% $ S \quad $ Function of values of a physical variable over time $ t_i $. \\
% $ t_i \quad $ An instance in time. \\
% $ L \quad $ A value. \\
% $ q \quad $ An $ L $-indexed question of the form $ S(t_i) = L $? \\
$ U \quad $ Set of possible histories of the universe. \\
$ u \quad $ A history of the universe in $ U $. \\ 
$ X \quad $ Setup function of an ID that maps $ U \rightarrow X(U) $. A binary question concerning $ \Gamma(u) $. \\
$ x \quad $ A binary question and a member of image $ X(U) $. \\ 
$ Y \quad $ Conclusion function of an ID that maps $ U \rightarrow \{-1, 1\} $. A binary answer of an ID for  $ X(u) = x $. \\ 
$ y \quad $ A single-valued answer, and member of image $ Y(U)  = \{0, 1\} $. \\ 
$ \Gamma \quad $ A function of the actual values of a physical variable over $U$, equivalent to $\Gamma(u) = S(t_i)(u)$.  \\
$ \gamma \quad $ Possible value of a physical variable, a member of the image $\Gamma(U)$. \\
$ \delta \quad $ Probe of any variable $V$ parameterized by $v \in V$ such that : 
	  \[ \delta_v (v') =
	  \begin{cases} 
       1 & \text{ if } v = v' \\
       -1 & \text{ otherwise } \\
      \end{cases}\] \\
$ \wp \quad $ Set of probes over $\Gamma(U)$. \\
$ \mathcal{D} = (X, Y) \quad $ An inference device, consisting of functions $ X $ and $ Y $. \\
$ \bar{F} \quad $ Inverse. Given a function $ F $ over $ U $, $F ^ {-1} = \bar{F} \equiv \{\{u : F(u) = f \} : f \in F(U) \} $. \\
$ > \quad $ Weak inference: a device $\mathcal{D}$ weakly infers $\Gamma$ \textit{iff} $ \forall \gamma \in \Gamma(U), \exists x \in X(U) $ s.t. $ \forall u \in U $, 

$ \quad X(u) = x \implies Y(u) = \delta_{\gamma}(\Gamma(u)) $.  \\
$ >> \quad $ Strong inference: a device $ (X_1, Y_1) $ strongly infers a device $ (X_2, Y_2) $ \textit{ iff } $\forall \delta \in \wp(Y_2) $ 

\quad and all $ x_2 $, $ \exists x_1 $ \textit{ such that } $ X_1 = x_1 \implies X_2 = x_2, Y_1 = \delta(Y_2) $. \\

\newpage
\textbf{Turing Machines} \\
% The quadruple you list isn't a set of rules - it's a quadruple. In fact, a TM is defined by (in your notation) by Q, p, A, S and a special halt state in Q, along with a rule taking Q x A -> Q x S, where we have to say what each element a of S would do to a semi-infinite string with current element s in A.

%Li and Vitanyi informally specify a Turing Machine (TM) according to a set of rules $(p, s, a, q)$ corresponding to a starting state, scanned input, action, and end state such that $p, q \in Q$, $s \in S = \{0, 1, B\}$, and $a \in A = \{0, 1, B, L, R\} $. Hence, a TM can be represented by a mapping between $ Q \times S \rightarrow S \times Q $. Note that $ Q $ is a finite set of internal states. Any two distinctive $(p, s, a, q)$ quadruples must differ in the first two elements, such that the Turing machine is deterministic. This definition may be useful for gaining intuition about Turing Machines but we use a more formal definition for our purposes: \\

 Arora and Barak denote a Turing Machine $ T $ as $ T = (\Gamma, Q, \delta) $ containing:
\begin{enumerate}
\item An \textit{alphabet} $ \Gamma $ of a finite set of symbols that $ T $'s tapes can contain. We assume that $ \Gamma $ contains a special blank symbol $ B $, start symbol $ S $, and the numbers 0 and 1. 
\item A finite set $ Q $ of possible states that $ T $'s register can be in. We assume that $ Q $ contains a special start state $ q_{s} $ and a special halt state $ q_{h} $. 
\item A transition function function $ \delta : Q \times \Gamma^{k} \rightarrow Q \times \Gamma^{k - 1} \times \{L, S, R\}^{k} $, where $ k \geq 2$, describing the rules $ T $ use in performing each step. The set $\{L ,S, R\}$ denote the actions \textit{Left, Stay,} and \textit{Right}, respectively. 
\end{enumerate}

Suppose $ T $ is in state $ q \in Q $ and $ (\sigma_1, \sigma_2, \dots, \sigma_k) $ are the symbols on the $ k $ tapes. Then $ \delta(q, (\sigma_1, \dots, \sigma_k)) = (q', (\sigma_{2}^{'}, \dots, \sigma_{k}^{'}), z) $ where $ z \in \{L, S, R\}^k $ and at the next step the $ \sigma $ symbols in the last $ k - 1 $ tapes will be replaced by the $ \sigma' $ symbols, the machine will be in state $ q $, and the $ k $ heads will move \textit{Left, Right} or \textit{Stay}. \\

\textit{Remark}: $\Gamma$ can be reduced to $ \B = \{0, 1\} $ and $ k $ can be reduced to $ 1 $ without loss of computational power.

% The goal is to show how, for countably infinite U, for an *arbitrary* given TM (not some particular one with a single internal state), design an ID that strongly infers *the entire map* taking the starting instantaneous description (ID) of the TM to the halting ID, or to a special "never halt" symbol, as determined by that starting ID.

% Our goal is to infer the entire function implemented by the TM, not just a single iteration of the TM.

% \begin{center}
% \begin{tabular}{ c||c } 

% $ (p, s) $ & $ (a, q) $ \\ 
% \hline
% \hline
% $ (q, 0) $ & $ (0, q) $ \\ 
% \hline
% $ (q, 1) $ & $ (1, q) $ \\ 
% \hline
% $ (q, B) $ & $ (B, q) $ \\ 
 
% \end{tabular}
% \end{center}

% Cannot have two separate values of X(u) for the same u. Similarly for Y(u).

% I'm not sure how you chose this Gamma.  Is it the rule for a single iteration of the TM?

\pagebreak


\textbf{Inference of Turing Machines} \\
\\
\textbf{Theorem} \quad \textit{Every Turing Machine can be weakly inferred by an inference device.}\\
\textbf{Proof} \quad Recall the definition of weak inference: $$ \mathcal{D} > \Gamma \textit{ iff } \forall \gamma \in \Gamma(U), \exists x \in X(U) \text{ such that } \forall u \in U, X(u) = x \implies Y(u) = \delta_{\gamma}(\Gamma(u)) $$ \\
\\
\textbf{Theorem} \quad \textit{Every Turing Machine can be strongly inferred by an inference device.} \\
\textbf{Proof} \quad Recall the definition of strong inference:
$$ (X_1, Y_1) >> (X_2, Y_2) \textit{ iff } \forall \delta \in \wp(Y_2)\text{ and all } x_2,  \exists x_1 \textit{ such that } X_1 = x_1 \implies X_2 = x_2, Y_1 = \delta(Y_2) $$


% Write theorem for any Turing Machine encode it as gamma such that there is an inference device that can weakly infer that function

% Encode not the update rule but the entire map 

% Use countable spaces
% Gamma has to encode input string, output string or nonhalting








\end{document}